\chapter{\solutionSetsTitle}

Algebra problems can have multiple solutions. For example $x(x-1)=0$ has  two solutions: $0$ and $1$. By contrast, equations of the form $Ax=b$ with $A$ a linear operator have have the following property.\\



If $A$ is a linear operator and $b$ is a known then $Ax=b$ has either
\begin{enumerate}
\item One solution
\item  No solutions
\item Infinitely many solutions
\end{enumerate}


\section{The Geometry of Solution Sets: Hyperplanes}
Consider the following algebra problems and their solutions

\begin{enumerate}
\item $6x=12$, one solution: $2$
\item $0x=12$, no solution
\item $0x=0$, one solution for each number: $x$
\end{enumerate}
In each case the linear operator is a $1\times 1$ matrix. In the first case, the linear operator is invertible. 
In the other two cases it is not. 
In the first case, the solution set is a point on the number line, in the third case the solution set is the whole number line.

Lets examine similar situations with larger matrices.
\begin{enumerate}
\item
$\begin{pmatrix}
6	&0 	\\
0 	&2 	
\end{pmatrix} 
\begin{pmatrix}
 x \\ 
y 
\end{pmatrix} 
=
\begin{pmatrix}
12 \\ 
6
\end{pmatrix}$, one solution: 
$\begin{pmatrix}
2 \\ 
3
\end{pmatrix}$
%\\linear operator is invertible

\item 
$\begin{pmatrix}
1	&3 	\\
0 	&0 	
\end{pmatrix} 
\begin{pmatrix}
 x \\ 
y 
\end{pmatrix} 
=
\begin{pmatrix}
4 \\ 
1 
\end{pmatrix}$, no solutions
%not in the range of the linear operator

\item 
$\begin{pmatrix}
1	&3 	\\
0 	&0 	
\end{pmatrix} 
\begin{pmatrix}
 x \\ 
y 
\end{pmatrix} 
=
\begin{pmatrix}
4 \\ 
0
\end{pmatrix} $, one solution for each number $y$: 
$\begin{pmatrix}
4-3y \\ 
y
\end{pmatrix} $

\item 
$\begin{pmatrix}
0	&0 	\\
0 	&0 	
\end{pmatrix} 
\begin{pmatrix}
 x \\ 
y 
\end{pmatrix} 
=
\begin{pmatrix}
0 \\ 
0
\end{pmatrix} $, one solution for each pair of numbers $x,y$:
$\begin{pmatrix}
x\\ 
y
\end{pmatrix} $
\end{enumerate}
Again, in the first case the linear operator is invertible while in the other cases it is not. When the operator is not invertible the solution set can be empty, a line in the plane or the plane itself.


For a system of equations with $r$ equations and $k$ veriables, one can have a number of different outcomes.  For example, consider the case of $r$ equations in three variables.  Each of these equations is the equation of a plane in three-dimensional space.  To find solutions to the system of equations, we look for the common intersection of the planes (if an intersection exists).  Here we have five different possibilities:

\begin{enumerate}
\item \textbf{Unique Solution.}  The planes have a unique point of intersection.

\item \textbf{No solutions.}  Some of the equations are contradictory, so no solutions exist.

\item \textbf{Line.}  The planes intersect in a common line; any point on that line then gives a solution to the system of equations.

\item \textbf{Plane.}  Perhaps you only had one equation to begin with, or else all of the equations coincide geometrically.  In this case, you have a plane of solutions, with two free parameters.

\videoscriptlink{solution_sets_for_systems_of_linear_equations_planes.mp4}{Planes}{solution_sets_for_systems_of_linear_equations_planes}

\item \textbf{All of $\mathbb{R}^3$.}  If you start with no information, then any point in $\mathbb{R}^3$ is a solution.  There are three free parameters.
\end{enumerate}

In general, for systems of equations with $k$ unknowns, there are $k+2$ possible outcomes, corresponding to the possible numbers (i.e. $0,1,2,\dots,k$) of free parameters in the solutions set plus the possibility of no solutions.  These types of ``solution sets''\index{Solution set} are ``hyperplanes''\index{Hyperplane}, generalizations of planes the behave like planes in $\mathbb{R}^3$ in many ways.

\videoscriptlink{solution_sets_for_systems_of_linear_equations_overview.mp4}{Pictures and Explanation}{solution_sets_for_systems_of_linear_equations_overview}

\vspace{3mm}

\reading{4}{1}
%\begin{center}\href{\webworkurl ReadingHomework4/1/}{Reading homework: problem 4.1}\end{center}



\section{Particular Solution $+$ Homogeneous solutions }

In the \hyperlink{standard approach}{standard approach}, variables corresponding to columns that do not contain a pivot (after going to reduced row echelon form) are \emph{free}.  
We called them non-pivot variables. 
They index elements of the solutions set by acting as coefficients of vectors.
%In this way the number of non-pivot columns determines (in part) the size of the solution set.  
%We can denote them with dummy variables $\mu_1, \mu_2, \ldots$. 

\begin{example} (Non-pivot columns determine terms of the solutions)
$$\begin{pmatrix}
1 &  0 & 1 & -1 \\ 
 0 & 1 & -1& 1  \\
 0 &0   & 0  & 0 \\
\end{pmatrix}
\colvec{x_1\\x_2\\x_3\\x_4} 
=
\colvec{1\\1\\0} 
\Leftrightarrow
\left\{
\begin{array}{lcr}
	1x_1 +0x_2+ 1x_3 - 1x_4 & = 1 \\
	0x_1 +1x_2 - 1x_3 + 1x_4 & = 1 \\
	0x_1 +0x_2 + 0x_3 + 0x_4 & = 0 
\end{array}
     \right.
$$
Following the standard approach, express the pivot variables in terms of the non-pivot variables and add ``freebee equations". Here $x_3$ and $x_4$ are non-pivot variables.  
\begin{eqnarray*}
\left.
\begin{array}{rcl}
	x_1 & = &1 -x_3+x_4 \\
	x_2 & = &1 +x_3-x_4 \\
	x_3 & = &\phantom{1+~\,}x_3\\
	x_4 & =&\phantom{1+x_3+~}x_4         
\end{array}
     \right\}
     \Leftrightarrow
\colvec{x_1\\x_2\\x_3\\x_4} 
= \colvec{1\\1\\0\\0} + x_3\colvec{-1\\1\\1\\0} + x_4\colvec{1\\-1\\0\\1}
\end{eqnarray*}
The preferred way to write a solution set is with set notation\index{Solution set!set notation}.  \[S = \left\{\colvec{x_1\\x_2\\x_3\\x_4} = \colvec{1\\1\\ 0\\0 } + \mu_1 \colvec{-1\\1\\1\\0 }  + \mu_2  \colvec{1\\-1\\ 0 \\1 } : \mu_1,\mu_2\in  {\mathbb R} \right\} \]
Notice that the first two components of the second two terms come from the non-pivot columns
Another way to write the solution set is
\[S= \left\{  X_0 + \mu_1 Y_1 + \mu_2 Y_2   : \mu_1,\mu_2 \in  {\mathbb R}   \right\} \]
where 
\[X_0= \colvec{1\\1\\0 \\0 }, Y_1=\colvec{-1\\1\\1\\0 } , Y_2= \colvec{1\\-1\\0 \\1 }
\]
\end{example}
Here $X_0$ is called a particular solution while $Y_1$ and $Y_2$ are called homogeneous solutions. 



\section{Linearity and these parts}
%
%\begin{definition}   A function $f$ is \emph{linear}\index{Linear!function} if 
%for any vectors $X,Y$  in the domain of $f$, and any scalars $\alpha,\beta$ 
%\[f(\alpha X + \beta Y) = \alpha f(X) + \beta f(Y) \,.\]
%\end{definition}

%
%
%\begin{example}
%\hypertarget{solution_sets_for_systems_of_linear_equations_concrete_example}{Consider our example system above with} 
%\[
%M=    \begin{pmatrix}
%      1  & 0  & 1 & -1  \\
%       0  & 1 & -1 & 1  \\
%        0 &0   & 0  & 0    \\
%    \end{pmatrix} \, ,\quad
%X= \colvec{x_1\\x_2\\x_3\\x_4} \mbox{ and } Y=\colvec{y_1\\y_2\\y^3 \\y^4 }\, ,
%\]
%and take for the function of vectors
%$$
%f(X)=MX\, .
%$$
%Now let us check the linearity property for $f$. 
%The property needs to hold for {\it any} scalars $\alpha$ and $\beta$, so for simplicity
%let us concentrate first on the case $\alpha=\beta=1$. This means that we need to
%compare the following two calculations:
%\begin{enumerate}
%\item First add $X+Y$, then compute $f(X+Y)$.
%\item First compute $f(X)$ and $f(Y)$, then compute the sum $f(X)+f(Y)$.
%\end{enumerate}
%The second computation is slightly easier:
%$$
%f(X) = MX 
%    =\colvec{x_1+x_3-x_4\\x_2-x_3+x_4\\0}\mbox{ and }
%f(Y) = MY   
%    =\colvec{y_1+y_3-y_4\\y_2-y_3+y_4\\0}\, ,
%$$
%(using our result above). Adding these gives
%$$
%f(X)+f(Y)=\colvec{x_1+x_3-x_4+y_1+y_3-y_4\\[1mm]x_2-x_3+x_4+y_2-y_3+y_4\\[1mm]0}\, .
%$$
%Next we perform the first computation beginning with:
%$$
%X+Y=\colvec{x_1 + y_1\\x_2+y_2\\ x_3+y_3\\ x_4+y_4}\, ,
%$$
%from which we calculate
%$$
%f(X+Y)=\colvec{x_1+y_2+x_3+y_3-(x_4+y_4)\\[1mm] x_2+y_2-(x_3+y_3)+x_4+y_4\\[1mm]0}\, .
%$$
%Distributing the minus signs and remembering that the order of adding numbers like $x_1,x_2,\ldots$ 
%does not matter, we see that the two computations give exactly the same answer.
%
%Of course, you should complain that we took a special choice of $\alpha$ and $\beta$.
%Actually, to take care of this we only need to check that $f(\alpha X)=\alpha f(X)$.
%It is your job to explain this in  \hyperref[linear]{Review Question~\ref*{linear}}
%\end{example}
%
%\noindent
%Later we will show that matrix multiplication is always linear.  Then we will know that:
With the previous example in mind, lets say that the matrix equation $MX=V$ has  solution set  $\{ X_0 + \mu_1 Y_1 + \mu_2 Y_2):\mu_1,\mu_2 \in {\mathbb R} \}$.
Recall from \hyperlink{{Matrices are linear operators}}{earlier} that matrices are linear.
%\[M(\alpha X + \beta Y) = \alpha MX + \beta MY\]
%
%Then 
%
%the two equations 
Thus 
$$M( X_0 + \mu_1 Y_1 + \mu_2 Y_2)  = MX_0 + \mu_1MY_1 + \mu_2MY_2 =V$$
for \emph{any} $\mu_1, \mu_2 \in \mathbb{R}$. 
Choosing $\mu_1=\mu_2=0$, we obtain 
$$MX_0=V\, .$$  
This is why $X_0$ is an example of a  \emph{particular solution}\index{Particular solution!an example}.

%Given a particular solution to the system, we can then deduce that $\mu_1MY_1 + \mu_2MY_2 = 0$.  
Setting $\mu_1=1, \mu_2=0$, and using the particular solution $MX_0=V$, we obtain 
$$MY_1=0\, .$$ 
Likewise, setting $\mu_1=0, \mu_2=1$, we obtain $$MY_2=0\, .$$
Here $Y_1$ and $Y_2$ are examples of what are called  \emph{homogeneous} solutions\index{Homogeneous solution!an example} to the system.
They {\it do not} solve the original equation $MX=V$, but instead its associated 
{\it homogeneous  equation}\index{homogeneous equation} $M Y =0$.

One of the fundamental lessons of linear algebra: the  solution set to $Ax=b$ with $A$ a linear operator consists of a particular solution plus homogeneous solutions.

\begin{center}
\shabox{
general solution $=$ particular solution $+$ homogeneous solutions.}
\end{center}

\begin{example}
Consider the matrix equation of the previous example. It has  solution set
\[S = \left\{\colvec{x_1\\x_2\\x_3\\x_4} = \colvec{1\\1\\0 \\0 } + \mu_1 \colvec{-1\\1\\1\\0 } + \mu_2 \colvec{1\\-1\\ 0\\1 } \right\} \]
Then $MX_0=V$ says that $\colvec{x_1\\x_2\\x_3\\x_4} = 
\colvec{1\\1\\0 \\ 0}$ solves the original matrix equation, which is certainly true, but this is not the only solution.

$MY_1=0$ says that $\colvec{x_1\\x_2\\x_3\\x_4} = \colvec{-1\\1\\1\\ 0}
$ solves the homogeneous equation.

\vspace{2mm}

$MY_2=0$ says that $\colvec{x_1\\x_2\\x_3\\x_4} = 
\colvec{1\\-1\\0 \\1}$ solves the homogeneous equation.

\vspace{2mm}

\noindent
Notice how adding any multiple of a homogeneous solution to the particular solution yields another particular solution.
\end{example}

%\begin{definition}
%Let $M$ a matrix and $V$ a vector.  Given the linear system $MX=V$, we call $X_0$ a \emph{particular solution}\index{Particular solution} if $MX_0=V$.  We call $Y$ a \emph{homogeneous solution} if $MY=0$.  
%The linear system 
%$$MX=0$$ is called the (associated) \emph{homogeneous system}\index{Homogeneous system}.
%\end{definition}
%
%If $X_0$ is a particular solution, then the general solution\index{General solution} to the system is\footnote{The notation \(S=\{X_0+Y : MY=0\}\) is read, ``\(S\) is the set of all \(X_0+Y\) such that \(MY=0,\)'' and means exactly that. Sometimes a pipe \(|\) is used instead of a colon.}:
%
%\[S= \{X_0+Y : MY=0 \} \]

\reading{4}{2}
%\begin{center}\href{\webworkurl ReadingHomework4/2/}{Reading homework: problem 4.2}\end{center}

%\section*{References}
%
%Hefferon, Chapter One, Section I.2
%\\
%Beezer, Chapter SLE, Section TSS
%\\
%Wikipedia, \href{http://en.wikipedia.org/wiki/System_of_linear_equations}{Systems of Linear Equations}


%\section{The size of solution sets vs size of homogeneous solution set}


\section{Review Problems}




\begin{enumerate}

\item While performing  Gaussian elimination on these augmented matrices write the full system of equations describing the new rows in terms of the old rows above each equivalence symbol as in  \hyperlink{Keeping track of EROs with equations between rows}{Example}~\ref{Rsystem}. 
$$
\begin{amatrix}{2} 
2 & 2 & 10 \\
1 & 2 & 8 \\
\end{amatrix}
,~
\begin{amatrix}{3} 
1 & 1 & 0 & 5 \\
1 & 1 & \!\!-1& 11 \\
-1 & 1 & 1 & -5 \\ 
\end{amatrix}
$$

%%%%%%%%%%%%%%%%%%%

\item Solve the vector equation by applying ERO matrices to each side of the equation to perform elimination. Show each matrix explicitly as in \hyperlink{Undoing}{Example~\ref{slowly}}.

\begin{eqnarray*}
\begin{pmatrix}
3	&6 	&2 \\ %-3
5 	&9 	&4 \\ %1
2	&4	&2 \\ %0
\end{pmatrix} 
\begin{pmatrix}
 x \\ 
y \\
z 
\end{pmatrix} 
=
\begin{pmatrix}
-3 \\ 
1  \\
0  \\
\end{pmatrix} 
\end{eqnarray*}

%%%%%%%%%%%%%%%%%%%

\item Solve this vector equation by finding the inverse of the matrix through $(M|I)\sim (I|M^{-1})$ and then applying $M^{-1}$ to both sides of the equation. 
\begin{eqnarray*}
\begin{pmatrix}
2	&1 	&1 \\ %9
1 	&1 	&1 \\ %6
1	&1	&2 \\ %7
\end{pmatrix} 
\begin{pmatrix}
 x \\ 
y \\
z 
\end{pmatrix} 
=
\begin{pmatrix}
9 \\ 
6  \\
7  \\
\end{pmatrix} 
\end{eqnarray*}


%%%%%%%%%%%%%%%%%%%

\item Follow the method of  \hyperlink{elldeeeww}{Examples~\ref{factorize} and~\ref{factorizes}} to find the $LU$ and $LDU$ factorization of 
\begin{eqnarray*}
\begin{pmatrix}
3	&3 	&6 \\ %0 %2
3 	&5 	&2 \\ %1 %1
6	&2	&5 \\ %0 %1
\end{pmatrix} .
\end{eqnarray*}



%%%%%%%%%%%%%%%%%%%%

\item 
Multiple matrix equations with the same matrix can be solved simultaneously. 
\begin{enumerate}
\item Solve both systems by performing elimination on just one augmented matrix.
\begin{eqnarray*}
\begin{pmatrix}
2	&-1 	&-1 \\ %0 %2
-1 	&1 	&1 \\ %1 %1
1	&-1	&0 \\ %0 %1
\end{pmatrix} 
\begin{pmatrix}
 x \\ 
y \\
z 
\end{pmatrix} 
=
\begin{pmatrix}
0\\ 
1  \\
0  \\
\end{pmatrix} 
,~
\begin{pmatrix}
2	&-1 	&-1 \\ %0 %2
-1 	&1 	&1 \\ %1 %1
1	&-1	&0 \\ %0 %1
\end{pmatrix} 
\begin{pmatrix}
 a \\ 
b \\
c 
\end{pmatrix} 
=
\begin{pmatrix}
2\\ 
1  \\
1  \\
\end{pmatrix} 
\end{eqnarray*}
\item Give an interpretation of the columns of $M^{-1}$ in $(M|I)\sim (I|M^{-1})$ in terms of solutions to certain systems of linear equations.
\end{enumerate}

%%%%%%%%%%%%%%%%%%%%%%%%

\item How can you convince your fellow students to never make this mistake?
\begin{eqnarray*}
\begin{amatrix}{3} 
1 & 0 & 2 & 3 \\ 
0 & 1 & 2& 3 \\
2 & 0 & 1 & 4 \\
\end{amatrix} 
& 
\stackrel{R_1'=R_1+R_2}{
\stackrel{R_2'=R_1-R_2}{ 
\stackrel{\ R_3'= R_1+2R_2}{\sim}}}
&
\begin{amatrix}{3} 
1 & 1 & 4 & 6 \\
1 & \!\!-1 & 0& 0 \\
1 & 2 & 6 & 9 
\end{amatrix}
\end{eqnarray*}

\item Is $LU$ factorization of a matrix unique?  Justify your answer.


\item[$\infty$.] If you randomly create a matrix by picking numbers out of the blue, it will probably be difficult to perform elimination or factorization; fractions and large numbers will probably be involved. To invent simple problems it is better to start with a simple answer:
\begin{enumerate}
\item Start with any augmented matrix in RREF. Perform EROs to make most of the components non-zero. Write the result on a separate piece of paper and give it to your friend. Ask that friend to find RREF of the augmented matrix you gave them. Make sure they get the same augmented matrix you started with.  
\item Create  an upper triangular matrix $U$ and a lower triangular matrix~$L$ with only $1$s on the diagonal. Give the result to a friend to factor into $LU$ form. 
\item Do the same with an $LDU$ factorization. 
\end{enumerate}
\end{enumerate}

\phantomnewpage




\newpage
