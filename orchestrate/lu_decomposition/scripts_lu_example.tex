
\subsection*{Another $LU$ Decomposition Example}

%%%Insert this to get the typewriter font so it looks like a real movie script
{\ttfamily
\fontdimen2\font=0.4em
\fontdimen3\font=0.2em
\fontdimen4\font=0.1em
\fontdimen7\font=0.1em
\hyphenchar\font=`\-


\hypertarget{scripts_lu_decomposition_example}{Here we will perform} an $LU$ decomposition on the matrix
\[
M = \begin{pmatrix}
1 & 7 & 2 \\
-3 & -21 & 4 \\
1 & 6 & 3
\end{pmatrix}
\]
following the \hyperref[finding_LU_decomp]{procedure outlined in Section~\ref*{finding_LU_decomp}}. So initially we have $L_1 = I_3$ and $U_1 = M$, and hence
\begin{align*}
L_2 & = \begin{pmatrix}
1 & 0 & 0 \\
-3 & 1 & 0 \\
1 & 0 & 1
\end{pmatrix}
& U_2 & = \begin{pmatrix}
1 & 7 & 2 \\
0 & 0 & 10 \\
0 & -1 & -1
\end{pmatrix}.
\end{align*}
However we now have a problem since $0 \cdot c = 0$ for any value of $c$ since we are working over a field, but we can quickly remedy this by swapping the second and third rows of $U_2$ to get $U_2^{\prime}$ and note that we just interchange the corresponding rows all columns left of and including the column we added values to in $L_2$ to get $L_2^{\prime}$. Yet this gives us a small problem as $L_2^{\prime} U_2^{\prime} \neq M$; in fact it gives us the similar matrix $M^{\prime}$ with the second and third rows swapped. In our original problem $MX = V$, we also need to make the corresponding swap on our vector $V$ to get a $V^{\prime}$ since all of this amounts to changing the order of our two equations, and note that this clearly does not change the solution. Back to our example, we have
\begin{align*}
L_2^{\prime} & = \begin{pmatrix}
1 & 0 & 0 \\
1 & 1 & 0 \\
-3 & 0 & 1
\end{pmatrix}
& U_2^{\prime} & = \begin{pmatrix}
1 & 7 & 2 \\
0 & -1 & -1 \\
0 & 0 & 10
\end{pmatrix},
\end{align*}
and note that $U_2^{\prime}$ is upper triangular. Finally you can easily see that
\[
L_2^{\prime} U_2^{\prime} = \begin{pmatrix}
1 & 7 & 2 \\
1 & 6 & 3 \\
-3 & -21 & 4
\end{pmatrix} = M^{\prime}
\]
which solves the problem of $L_2^{\prime} U_2^{\prime} X = M^{\prime} X = V^{\prime}$. (We note that as augmented matrices $( M^{\prime} | V^{\prime} ) \sim (M | V)$.)

} % Closing bracket for font

%\newpage
