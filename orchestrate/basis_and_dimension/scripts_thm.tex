
\subsection*{Proof Explanation}

%%%Insert this to get the typewriter font so it looks like a real movie script
{\ttfamily
\fontdimen2\font=0.4em
\fontdimen3\font=0.2em
\fontdimen4\font=0.1em
\fontdimen7\font=0.1em
\hyphenchar\font=`\-


\hypertarget{basis_and_dimension_thm}{Lets walk } through the proof of theorem~\ref{uniqvec}. We want to show that for $S=\{v_1, \ldots, v_n \}$ a basis for a vector space $V$, then every vector $w \in V$ can be written \emph{uniquely} as a linear combination of vectors in the basis $S$:
\[
w=c^1v_1+\cdots + c^nv_n.
\]

We should remember that since $S$ is a basis for $V$, we know two things
\begin{itemize}
\item $V = \spa S$
\item $ v_1, \ldots , v_n$ are linearly independent, which means that whenever we have 
$
a^1v_1+ \ldots + a^n v_n = 0
$
this implies that $a^i =0$ for all $i=1, \ldots, n$.
\end{itemize}
This first fact makes it easy to say that there exist constants $c^i$ such that $w=c^1v_1+\cdots + c^nv_n$. What we don't yet know is that these $c^1, \ldots c^n$ are unique.

 In order to show that these are unique, we will suppose that they are not, and show that this causes a contradiction. So suppose there exists a second set of constants $d^i$ such that 
$$w=d^1v_1+\cdots + d^nv_n\, .$$ 
For this to be a contradiction we need to have $c^i \neq d^i$ for some $i$. Then look what happens when we take the difference of these two versions of $w$:
\begin{eqnarray*}
0_V&=&w-w\\
&=&(c^1v_1+\cdots + c^nv_n)-(d^1v_1+\cdots + d^nv_n )\\
&=&(c^1-d^1)v_1+\cdots + (c^n-d^n)v_n. \\
\end{eqnarray*}
Since the $v_i$'s are linearly independent this implies that $c^i - d^i = 0$ for all $i$, this means that we cannot have $c^i \neq d^i$, which is a contradiction.



} % Closing bracket for font

%\newpage
