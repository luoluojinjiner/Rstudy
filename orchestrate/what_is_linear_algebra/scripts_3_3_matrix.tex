
\subsection{\whatIsTitle: $3 \times 3$ Matrix Example}

{\ttfamily
\fontdimen2\font=0.4em
\fontdimen3\font=0.2em
\fontdimen4\font=0.1em
\fontdimen7\font=0.1em
\hyphenchar\font=`\-

\hypertarget{scripts_what_is_linear_algebra_3_3_matrix}{Your friend places a jar} on a table and tells you that there is 65 cents in this jar with 7 coins consisting of quarters, nickels, and dimes, and that there are twice as many dimes as quarters. Your friend wants to know how many nickels, dimes, and quarters are in the jar.

We can translate this into a system of the following linear equations:
\begin{align*}
5n + 10d + 25q & = 65
\\ n + d + q & = 7
\\ d & = 2q
\end{align*}
Now we can rewrite the last equation in the form of $-d + 2q = 0$, and thus express this problem as the matrix equation
\[
\begin{pmatrix}
5 & 10 & 25 \\
1 & 1 & 1 \\
0 & -1 & 2
\end{pmatrix} \begin{pmatrix}n\\d\\q\end{pmatrix} = \begin{pmatrix}65\\7\\0\end{pmatrix}.
\]
or as an \hyperlink{augmented_matrix}{augmented matrix} (see also \hyperlink{script_gaussian_elimination_more}{this script on the notation}).
\[
\begin{pmatrix}
5 & 10 & 25 & \vline & 65\\
1 & 1 & 1 & \vline & 7 \\
0 & -1 & 2 & \vline & 0
\end{pmatrix}
\]
Now to solve it, using our original set of equations and by substitution, we have
\begin{align*}
5n + 20q + 25q = 5n + 45q & = 65
\\ n + 2q + q = n + 3q & = 7
\end{align*}
and by subtracting 5 times the bottom equation from the top, we get
\[
45q - 15q = 30q = 65 - 35 = 30
\]
and hence $q = 1$. Clearly $d = 2$, and hence $n = 7 - 2 - 1 = 4$. Therefore there are four nickels, two dimes, and one quarter.

} % Closing brace for font

\newpage
