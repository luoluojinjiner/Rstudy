
\subsection*{Linear systems as spanning sets}

{\ttfamily
\fontdimen2\font=0.4em
\fontdimen3\font=0.2em
\fontdimen4\font=0.1em
\fontdimen7\font=0.1em
\hyphenchar\font=`\-

\hypertarget{scripts_subspaces_and_spanning_sets_example}{Suppose that we were given a} set of linear equations $l^j(x^1, x^2, \dotsc, x^n)$ and we want to find out if $l^j(X) = v^j$ for all $j$ for some vector $V = (v^j)$. We know that we can express this as the matrix equation
\[
\sum_i l^j_i x^i = v^j
\]
where $l^j_i$ is the coefficient of the variable $x^i$ in the equation $l^j$. However, this is also stating that $V$ is in the span of the vectors $\{ L_i \}_i$ where $L_i = (l^j_i)_j$. For example, consider the set of equations
\begin{align*}
2 x + 3 y - z & = 5
\\ -x + 3y + z & = 1
\\ x + y - 2 z & = 3
\end{align*}
which corresponds to the matrix equation
\[
\begin{pmatrix}
2 & 3 & -1 \\
-1 & 3 & 1 \\
1 & 1 & -2
\end{pmatrix} \begin{pmatrix} x \\ y \\ z \end{pmatrix} = \begin{pmatrix} 5 \\ 1 \\ 3 \end{pmatrix}.
\]
We can thus express this problem as determining if the vector
\[
V = \begin{pmatrix} 5 \\ 1 \\ 3 \end{pmatrix}
\]
lies in the span of
\[
\left\{ \begin{pmatrix} 2 \\ -1 \\ 1 \end{pmatrix}, \begin{pmatrix} 3 \\ 3 \\ 1 \end{pmatrix}, \begin{pmatrix} -1 \\ 1 \\ -2 \end{pmatrix} \right\}.
\]

} % Closing brace for the font

%\newpage
