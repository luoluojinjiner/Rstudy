


\begin{enumerate}
\item Show that the pair of conditions:
\begin{equation}\label{aleph}\left\{
\begin{array}{l}L(u+v) = L(u)+L(v)\\[1mm]L(cv) = cL(v) \end{array}\right.\tag{1}
\end{equation}
(valid for all vectors $u,v$ and any scalar $c$) is equivalent to the single condition:
\begin{equation}\label{null}L(ru + sv) = rL(u) + sL(v)\, ,\tag{2}\end{equation}
(for all vectors $u,v$ and any scalars $r$ and $s$).
Your answer should have two parts.  Show that~(\ref{aleph})~$\Rightarrow$~(\ref{null}), and then show that~(\ref{null})~$\Rightarrow$~(\ref{aleph}).

\phantomnewpage

\item If $f$ is a linear function of one variable, then how many points on the graph of the function are needed to specify the function? Give an explicit expression for $f$ in terms of these points. (You might want to look up the definition of a graph before you make any assumptions about the function.)

\item 
\begin{enumerate}\item If $p\colvec{1\\2}=1$ and $p\colvec{2\\4}=3$ is it possible that $p$ is a linear function? 
\item
 If $Q(x^2)=x^3$ and $Q(2x^2)=x^4$ is it possible that $Q$ is a linear function from polynomials to polynomials? 
\end{enumerate}

\phantomnewpage

\item If $f$ is a linear function such that 
$$f\colvec{1\\2}=0{\rm ,~and~} f\colvec{2\\3}=1\, ,$$ 
then what is $f\colvec{x\\y}$?

\phantomnewpage

\item \label{polyprob}Let $P_n$ be the space of polynomials of degree $n$ or less in the variable $t$.  Suppose $L$ is a linear transformation from $P_2 \rightarrow P_3$ such that
$L(1) = 4$, $L(t)=t^3$, and $L(t^2) = t-1$.
\begin{enumerate}
\item Find $L(1+t+2t^2)$.
\item Find $L(a+bt+ct^2)$.
\item Find all values $a,b,c$ such that $L(a+bt+ct^2)=1+3t+2t^3$.
\end{enumerate}

\Videoscriptlink{linear_transformations_hint.mp4}{Hint}{scripts_linear_transformations_hint}

\phantomnewpage
  
\item Show that the operator ${\cal I}$ that maps $f$ to the function ${\cal I}f$ defined by ${\cal I}f(x):=\int_0^xf(t)dt$ is a linear operator on the space of continuous functions. 
%How many inputs completely.
% is a linear transformation on the vector space of polynomials.  What would a matrix for integration look like? Be sure to think about what to do with the constant of integration.
%What? 
%\videoscriptlink{linear_transformations_deriv_int}{Finite degree example}{scripts_linear_transformations_deriv_int}


\phantomnewpage

\item Let $z \in \mathbb{C}$. Recall that  $z = x + iy$ for some $x,y \in \mathbb{R}$, and we can form the \emph{complex conjugate} of $z$ by taking $\overline{z} = x - iy$. 
%(note that this is unique since $\overline{\overline{z}} = z$). 
The function $c \colon \mathbb{R}^2 \rightarrow \mathbb{R}^2$ which sends $(x, y) \mapsto (x, -y)$ agrees with complex conjugation.
\begin{enumerate}
\item Show that $c$ is a linear map over $\mathbb{R}$ ({\it i.e.} scalars in $\mathbb{R}$).

\item Show that $\overline{z}$ is not linear over $\mathbb{C}$.
\end{enumerate}

\end{enumerate}

\phantomnewpage