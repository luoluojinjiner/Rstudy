%%%%%%%%%%%%%%%%%% Latex %%%%%%%%%%%%%%%%%%%%%
\documentclass[12pt]{article}

\usepackage{amsmath}
\usepackage{shadow}
%\usepackage{bbold}
\usepackage{amssymb}
\usepackage{slashed}
%\usepackage{graphics}
\usepackage{graphicx}
%\usepackage{pstricks,pst-node,pst-tree}


%Sets
\def\A{{\mathbb{A}}}
\def\C{{\mathbb{C}}}
\def\F{{\cal F}}
\def\H{{\mathbb{H}}}
\def\O{{\cal O}}
\def\Q{{\mathbb{Q}}}
\def\R{{\mathbb{R}}}
\def\Real{\mathbb R}
%\def\R{{\cal R}}
\def\Z{{\mathbb{Z}}}

%greek
\def\a{\alpha}
\def\b{\beta}
\def\de{\delta} 
\def\De{\Delta}
\def\ga{\gamma} 
\def\g{\gamma}
\def\si{\sigma} 
\def\Si{{\Sigma}}
\def\e{\epsilon}
\def\eps{\epsilon}
\def\La{\Lambda}
\def\r{\rho}
\def\Th{\Theta}
\def\th{\theta} 
\def\k{\kappa}
\def\l{\lambda}
\def\m{\mu}
\def\n{\nu}
\def\s{\sigma}
\def\t{\tau}
\def\z{\zeta}

%equation 
\def\be{\begin{equation}}
\def\ee{\end{equation}}
\def\bea{\begin{eqnarray}}
\def\eea{\end{eqnarray}} 
\def\nn{\nonumber}

%matricies
\def\ba{\left(\begin{array}{cc}}
\def\ea{\end{array}\right) }
\def\baa{\left(\begin{array}{ccc}}
\def\eaa{\end{array}\right) }
\def\baaa{\left(\begin{array}{cccc}}
\def\eaaa{\end{array}\right) }

%vectors
\def\v{\vec}
\def\bv{\left(\begin{array}{c}}
\def\ev{\end{array}\right) }


%quantum operators
%\def\d{{\bf d}}
\def\D{{\bf {\cal D}}}
\def\Del{\bm \delta}
\def\Delb{\bar{\bm \delta}}
\def\DIV{{\bf div}}
\def\G{{\bf g}}
\def\GRAD{{\bf grad}}
\def\N{{\bf N}}
\def\ord{{\bf ord}}
\def\Pa{\bm \partial}
\def\Partial{\bm \partial}
\def\Pab{\bar{\bm \partial}}
\def\TR{{\bf tr}}
\def\dsl{\slashed{\partial}} 
\def\psl{\slashed{p}} 

%misc
\def\la{\langle}
\def\ra{\rangle}
\def\scirc{\!{\scriptstyle \circ}}
\def\pa{\partial}
\def\bra{\langle} \def\ket{\rangle}
\def\ie{{i.e., }}
\def\eg{{e.g.\ }}
\def\cl{\centerline}
\def\noi{\noindent}
\def\f{\frac}
\def\LR{\Leftrightarrow}
\def\d{\frac{d}{dx}}

%\newcommand{\comment}[1]{{\bf [#1]}}
%\newcommand{\ul}[1]{\underline{#1}}
%\newcommand{\bm}[1]{\mbox{\boldmath $ #1 $}}

%%%%%%%%%%%%%%%%%%%%%%%%%%%%%%%%%%%

\begin{document}

\thispagestyle{empty}
~
\vspace{-3cm}

\begin{center}
\vspace{-1.5cm}
{\Large{\bf  
To do for Linear, The Book
}
 }  
% \\[4mm]
% {\sc \small Cherney}
\end{center}



%Know how to solve systems of linear equations
\begin{enumerate}

\item every $L$ has at least one eigenvector... where is this?

\item some chapters do not have reading problems. Cherney's fault. Wally's job.

\item 7.2.5 is repeaty, mention cramer

\item 7.1 is repeaty, put the ordered set= row of columns notation earlier

\item reinstate references....? with new refs?

\item reinstate scripts, 

\item give scripts to cherney's vids ()

\item Improve links to webwork at the beginning of review problem sections

\item Chapter 1 should have a "proof that matrices are linear" video. 

\item det chapter should have n-volume and parallelotope discussions. Set notation for parallelepipeds and for spheres.

\item A common point of confusion for students: span takes in sets and gives out vector spaces, which are sets. Linear dependence is a property of sets, not individual vectors. Find a way to get the to respect the technical term "set" early on. 

\item Replace low quality vids with higher quality. 
\begin{enumerate}
\item work from scripts when possible.
\end{enumerate}

\item give videos more descriptive titles and 

\item Chapters on SVD, Fourier transform, Lagrange transform, 

\item upload webwork to national library

\item create a webpage/ blog for availability and other marketing 

\item we need an introduction, C says ``there is kindergarten arithmetic,
linear algebra allows us to turn real world problems into this...''

\end{enumerate}


a sketch of an introduction, by (Cherney): 

Many linear algebra text books claim to be applications oriented or concept oriented. This book rejects that dichotomy. 
The most important concept presented in this book: real world problems can be reduced to elementary arithmetic problems when one can identify the objects in play as vectors and linear operators. 

In five points: 
\begin{enumerate}
\item Vectors can be represented as one column matrices through a choice of ordered bases for the vector space.
\item Linear functions can be represented as matrices through a choice of two ordered bases, one for the domain and one for the target (or codomain).
\item Once these representations have been obtained, elimination and other tools can be utilized to solve difficult problems. Indeed, it is only the procedure of setting up these representations which is difficult. 
\item Diagonalization is the process of finding a basis in which the resulting matrix equations is easiest. 
\item The kernel and range of matrices provide a clean way to understand the relationship between particular solutions and homogeneous solutions via the orthogonal decomposition 
$\ker M \oplus_\perp {\rm ran}\, M^T$. (Though a discussion of the cokernel of $M$ would be more honest.)
\end{enumerate}
















\end{document}