
\chapter{\diagSymMatTitle}\label{symmetricmatrices}

Symmetric matrices have many applications.  For example, if we consider the shortest distance between pairs of important cities, we might get a table like the following.
\[
\begin{array}{c|ccc}
 & \text{Davis} & \text{Seattle} 
& \text{San Francisco} \\ \hline
\text{Davis} & 0 & 2000 & 80 \\
\text{Seattle} & 2000 & 0 & 2010 \\
\text{San Francisco} & 80 & 2010 & 0
\end{array}
\]
Encoded as a matrix, we obtain
\[
M=\begin{pmatrix}
\mc{0} & \mc{2000} & \mc{80} \\
\mc{2000} & \mc{0} & \mc{2010} \\
\mc{80} & \mc{2010} & \mc{0}
\end{pmatrix}=M^T.
\]

\begin{definition}
A matrix $M$ is {\bf symmetric}\index{Symmetric matrix} if  $M^T=M.$
\end{definition}

One very nice property of symmetric matrices is that they always have real eigenvalues.  Review exercise~\ref{prob_real_eigenvalues} guides you through the general proof, but below is an example for $2\times 2$ matrices.

\begin{example}
For a general symmetric $2\times 2$ matrix, we have:
\begin{eqnarray*}
P_\lambda \begin{pmatrix} a & b \\ b& d \end{pmatrix}
 &=&
\det\begin{pmatrix}\lambda-a&\mc{-b}\\\mc{-b}&\lambda-d \end{pmatrix}\\[1mm]
&=& (\lambda-a)(\lambda-d)-b^2 \\[2mm]
&=& \lambda^2-(a+d)\lambda-b^2+ad\\[1mm]
\Rightarrow \lambda &=& \frac{a+d}{2}\pm \sqrt{b^2+\left(\frac{a-d}{2}\right)^2}.
\end{eqnarray*}
Notice that the discriminant $4b^2+(a-d)^2$ is always positive, so that the eigenvalues must be real.
\end{example}

Now, suppose a symmetric matrix $M$ has two distinct eigenvalues $\lambda \neq \mu$ and eigenvectors $x$ and $y$;
\[
Mx=\lambda x, \qquad My=\mu y.
\] 
Consider the dot product $x\dotprod y = x^Ty = y^Tx$ and calculate
\begin{eqnarray*}
x^TM y &=& x^T\mu y = \mu x\dotprod y, \text{ and }\\[3mm]
x^TM y &=& (y^TMx)^T \text{ (by transposing a $1\times 1$ matrix)}\\[1mm]
       &=& (y^T\lambda x)^T \\
       &=& (\lambda x\dotprod y)^T \\
             &=& \lambda x\dotprod y.
\end{eqnarray*}
Subtracting these two results tells us that:
\begin{eqnarray*}
0 &=& x^TMy-x^TMy=(\mu-\lambda)\,x\dotprod y.
\end{eqnarray*}
Since $\mu$ and $\lambda$ were assumed to be distinct eigenvalues, $\lambda-\mu$ is non-zero, and so $x\dotprod y=0$.  We have proved the following theorem.

\begin{theorem}
Eigenvectors of a symmetric matrix with distinct eigenvalues are orthogonal.
\end{theorem}

%\begin{center}\href{\webworkurl ReadingHomework23/1/}{Reading homework: problem \ref{symmetricmatrices}.1}\end{center}
\Reading{DiagonalizingSymmetricMatrices}{1}

\begin{example}
The matrix $M=\begin{pmatrix}2&1\\1&2\end{pmatrix}$
has eigenvalues determined by
\[
\det(M-\lambda I)=(2-\lambda)^2-1=0.
\] 
So the eigenvalues of $M$ are $3$ and $1$, and the associated eigenvectors turn out to be 
$\colvec{1\\1}$ and $\colvec{1\\-1}$.  It is easily seen that these eigenvectors are \hyperref[orthogonal]{orthogonal}; 
\[
\colvec{1\\1} \dotprod \colvec{1\\-1}=0.
\]
\end{example}

In \hyperlink{basisorthog}{chapter~\ref{orthonormalbases}} we saw that the matrix $P$ built from any orthonormal basis  $(v_1,\ldots, v_n )$
for ${\mathbb R}^n$ as its columns,
\[
P=\rowvec{v_1 & \cdots & v_n}\, ,
\]
was an orthogonal matrix. This means that 
\[
P^{-1}=P^T, \text{ or } PP^T=I=P^TP.
\]
Moreover, given any (unit) vector $x_1$, one can always find vectors $x_2, \ldots, x_n$ such that $(x_1,\ldots, x_n)$ is an orthonormal basis.  (Such a basis can be obtained using the~\hyperref[GramSchmidt]{Gram-Schmidt procedure}.)

Now suppose $M$ is a symmetric $n\times n$ matrix and $\lambda_1$ is an eigenvalue with eigenvector $x_1$ (this is always the case because every matrix has at least one eigenvalue--see Review Problem~\ref{atleastone}).  
Let $P$ be the square matrix of orthonormal column vectors 
\[
P=\rowvec{x_1 & x_2 & \cdots & x_n},
\]
While $x_1$ is an eigenvector for $M$, the others are not necessarily eigenvectors for $M$.  
Then
\[
MP=\rowvec{\lambda_1 x_1 & Mx_2 & \cdots & Mx_n}.
\]
But $P$ is an orthogonal matrix, so $P^{-1}=P^T$.  Then:
\begin{eqnarray*}
P^{-1}=P^T &=& \ccolvec{x_1^T\\ \vdots \\ x_n^T} \\[1mm]
\Rightarrow P^TMP &=& \begin{pmatrix}
  x_1^T\lambda_1x_1  & * & \cdots & *\\
  x_2^T\lambda_1x_1  & * & \cdots & *\\
  \mc\vdots             &   & & \mc\vdots\\
   x_n^T\lambda_1x_1 & * & \cdots & *\\
  \end{pmatrix}\\[2mm]
&=& \begin{pmatrix}
  \lambda_1  & * & \cdots & *\\
  \mc 0  & * & \cdots & *\\
 \mc \vdots             & *  & & \mc\vdots\\
  \mc 0 & * & \cdots & *\\
  \end{pmatrix}\\[2mm]
&=& \begin{pmatrix}
  \lambda_1  & 0 & \cdots & 0\\
  \mc 0          & & & \\
  \mc\vdots     & & \hat{M} & \\
  \mc0          & & & \\
  \end{pmatrix}\, .\\
\end{eqnarray*}
The last equality follows since $P^TMP$ is symmetric.  The asterisks in the matrix are where ``stuff'' happens; this extra information is denoted by $\hat{M}$ in the final expression.  We know nothing about $\hat{M}$ except that it is an $(n-1)\times (n-1)$ matrix and that it is symmetric.  But then, by finding an (unit) eigenvector for $\hat{M}$, we could repeat this procedure successively.  The end result would be a diagonal matrix with eigenvalues of $M$ on the diagonal. Again, we have proved a theorem: %we also need that every matrix has an eigenvector.

\begin{theorem}
Every symmetric matrix is similar to a diagonal matrix of its eigenvalues.  In other words,
\[
M=M^T \Leftrightarrow M=PDP^T
\]
where $P$ is an orthogonal matrix and $D$ is a diagonal matrix whose entries are the eigenvalues of $M$.
\end{theorem}

%\begin{center}\href{\webworkurl ReadingHomework23/2/}{Reading homework: problem \ref{symmetricmatrices}.2}
\Reading{DiagonalizingSymmetricMatrices}{2}
%\end{center}

To diagonalize a real symmetric matrix, begin by building an orthogonal matrix from an orthonormal basis of eigenvectors, as in the example below. 

\begin{example}
The symmetric matrix 
$$M=\begin{pmatrix}2&1\\1&2\end{pmatrix}\,  ,$$ has eigenvalues $3$ and $1$ with eigenvectors $\colvec{1\\1}$ and $\colvec{1\\-1}$ respectively.  After normalizing these eigenvectors, we  build the orthogonal matrix:
\[
P = \begin{pmatrix}
\frac{1}{\sqrt{2}} & \frac{1}{\sqrt{2}} \\[2mm]
\frac{1}{\sqrt{2}} & \frac{-1}{\sqrt{2}}
\end{pmatrix}\, .
\]
Notice that $P^TP=I$.  Then:
\[
MP = \begin{pmatrix}
\frac{3}{\sqrt{2}} & \frac{1}{\sqrt{2}} \\[2mm]
\frac{3}{\sqrt{2}} & \frac{-1}{\sqrt{2}}
\end{pmatrix} = 
\begin{pmatrix}
\frac{1}{\sqrt{2}} & \frac{1}{\sqrt{2}} \\[2mm]
\frac{1}{\sqrt{2}} & \frac{-1}{\sqrt{2}}
\end{pmatrix} \begin{pmatrix}
3 & 0 \\[2mm]
0 & 1
\end{pmatrix}.
\]
In short, $MP=PD$, so $D=P^TMP$.  Then $D$ is the diagonalized form of $M$ and $P$ the associated change-of-basis matrix from the standard basis to the basis of eigenvectors.
\end{example}

\Videoscriptlink{diagonalizing_symmetric_matrices_3by3_example.mp4}{ $3\times 3$ Example}{scripts_diagonalizing_symmetric_matrices_3by3_example}












%\section*{References}
%Hefferon, Chapter Three, Section V: Change of Basis
%\\
%Beezer, Chapter E, Section PEE, Subsection EHM
%\\
%Beezer, Chapter E, Section SD, Subsection D
%\\
%Wikipedia:
%\begin{itemize}
%\item \href{http://en.wikipedia.org/wiki/Symmetric_matrix}{Symmetric Matrix}
%\item \href{http://en.wikipedia.org/wiki/Diagonalizable_matrix}{Diagonalizable Matrix}
%\item \href{http://en.wikipedia.org/wiki/Similar_matrix}{Similar Matrix}
%\end{itemize}

\section{Review Problems}

{\bf Webwork:} 
\begin{tabular}{|c|c|}
\hline
Reading Problems & 
 \hwrref{DiagonalizingSymmetricMatrices}{1}, 
 \hwrref{DiagonalizingSymmetricMatrices}{2}, 
 \\
Diagonalizing a symmetric matrix &  \hwref{DiagonalizingSymmetricMatrices}{3}, \hwref{DiagonalizingSymmetricMatrices}{4}\\
   \hline
\end{tabular}







\begin{enumerate}

\item While performing  Gaussian elimination on these augmented matrices write the full system of equations describing the new rows in terms of the old rows above each equivalence symbol as in  \hyperlink{Keeping track of EROs with equations between rows}{Example}~\ref{Rsystem}. 
$$
\begin{amatrix}{2} 
2 & 2 & 10 \\
1 & 2 & 8 \\
\end{amatrix}
,~
\begin{amatrix}{3} 
1 & 1 & 0 & 5 \\
1 & 1 & \!\!-1& 11 \\
-1 & 1 & 1 & -5 \\ 
\end{amatrix}
$$

%%%%%%%%%%%%%%%%%%%

\item Solve the vector equation by applying ERO matrices to each side of the equation to perform elimination. Show each matrix explicitly as in \hyperlink{Undoing}{Example~\ref{slowly}}.

\begin{eqnarray*}
\begin{pmatrix}
3	&6 	&2 \\ %-3
5 	&9 	&4 \\ %1
2	&4	&2 \\ %0
\end{pmatrix} 
\begin{pmatrix}
 x \\ 
y \\
z 
\end{pmatrix} 
=
\begin{pmatrix}
-3 \\ 
1  \\
0  \\
\end{pmatrix} 
\end{eqnarray*}

%%%%%%%%%%%%%%%%%%%

\item Solve this vector equation by finding the inverse of the matrix through $(M|I)\sim (I|M^{-1})$ and then applying $M^{-1}$ to both sides of the equation. 
\begin{eqnarray*}
\begin{pmatrix}
2	&1 	&1 \\ %9
1 	&1 	&1 \\ %6
1	&1	&2 \\ %7
\end{pmatrix} 
\begin{pmatrix}
 x \\ 
y \\
z 
\end{pmatrix} 
=
\begin{pmatrix}
9 \\ 
6  \\
7  \\
\end{pmatrix} 
\end{eqnarray*}


%%%%%%%%%%%%%%%%%%%

\item Follow the method of  \hyperlink{elldeeeww}{Examples~\ref{factorize} and~\ref{factorizes}} to find the $LU$ and $LDU$ factorization of 
\begin{eqnarray*}
\begin{pmatrix}
3	&3 	&6 \\ %0 %2
3 	&5 	&2 \\ %1 %1
6	&2	&5 \\ %0 %1
\end{pmatrix} .
\end{eqnarray*}



%%%%%%%%%%%%%%%%%%%%

\item 
Multiple matrix equations with the same matrix can be solved simultaneously. 
\begin{enumerate}
\item Solve both systems by performing elimination on just one augmented matrix.
\begin{eqnarray*}
\begin{pmatrix}
2	&-1 	&-1 \\ %0 %2
-1 	&1 	&1 \\ %1 %1
1	&-1	&0 \\ %0 %1
\end{pmatrix} 
\begin{pmatrix}
 x \\ 
y \\
z 
\end{pmatrix} 
=
\begin{pmatrix}
0\\ 
1  \\
0  \\
\end{pmatrix} 
,~
\begin{pmatrix}
2	&-1 	&-1 \\ %0 %2
-1 	&1 	&1 \\ %1 %1
1	&-1	&0 \\ %0 %1
\end{pmatrix} 
\begin{pmatrix}
 a \\ 
b \\
c 
\end{pmatrix} 
=
\begin{pmatrix}
2\\ 
1  \\
1  \\
\end{pmatrix} 
\end{eqnarray*}
\item Give an interpretation of the columns of $M^{-1}$ in $(M|I)\sim (I|M^{-1})$ in terms of solutions to certain systems of linear equations.
\end{enumerate}

%%%%%%%%%%%%%%%%%%%%%%%%

\item How can you convince your fellow students to never make this mistake?
\begin{eqnarray*}
\begin{amatrix}{3} 
1 & 0 & 2 & 3 \\ 
0 & 1 & 2& 3 \\
2 & 0 & 1 & 4 \\
\end{amatrix} 
& 
\stackrel{R_1'=R_1+R_2}{
\stackrel{R_2'=R_1-R_2}{ 
\stackrel{\ R_3'= R_1+2R_2}{\sim}}}
&
\begin{amatrix}{3} 
1 & 1 & 4 & 6 \\
1 & \!\!-1 & 0& 0 \\
1 & 2 & 6 & 9 
\end{amatrix}
\end{eqnarray*}

\item Is $LU$ factorization of a matrix unique?  Justify your answer.


\item[$\infty$.] If you randomly create a matrix by picking numbers out of the blue, it will probably be difficult to perform elimination or factorization; fractions and large numbers will probably be involved. To invent simple problems it is better to start with a simple answer:
\begin{enumerate}
\item Start with any augmented matrix in RREF. Perform EROs to make most of the components non-zero. Write the result on a separate piece of paper and give it to your friend. Ask that friend to find RREF of the augmented matrix you gave them. Make sure they get the same augmented matrix you started with.  
\item Create  an upper triangular matrix $U$ and a lower triangular matrix~$L$ with only $1$s on the diagonal. Give the result to a friend to factor into $LU$ form. 
\item Do the same with an $LDU$ factorization. 
\end{enumerate}
\end{enumerate}

\phantomnewpage



%\newpage
