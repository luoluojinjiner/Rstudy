


\begin{enumerate}

\item A door factory can buy supplies in two kinds of packages, $f$ and $g$. 
The package $f$ contains $3$ slabs of wood, $4$ fasteners, and $6$ brackets. 
The package $g$ contains $5$ fasteners, $3$ brackets, and $7$ slabs of wood. 
\begin{enumerate}
\item Explain how to view the packages $f$ and $g$ as functions and list their inputs and outputs. \\
\item Choose  an ordering for  the 3 kinds of supplies and use this to  rewrite $f$ and $g$ as elements of $\Re^3$.\\
\item Let $L$ be a manufacturing process that takes as inputs supply packages and outputs two products (doors, and door frames). Explain how it  can be viewed as a function 
mapping one vector space into another. 
\item
Assuming that $L$ is linear and  $Lf$ is $1$ door and $2$ frames, and $Lg$ is $3$ doors and $1$ frame, find a matrix for $L$. Be sure to specify the basis vectors you used, both for the input and output vector space.
\end{enumerate}

%%%%%%%%%%%%%%%%%%%%%%%%%%%%%%%%%%%

\item You are designing a simple keyboard synthesizer with two keys. 
If you push the first key with intensity $a$ then the speaker moves in time as $a\sin(t)$.
If you push the second key with intensity $b$ then the speaker moves in time as $b\sin(2t)$. 
If the keys are pressed simultaneously, \\
\begin{enumerate}
\item describe the set of all sounds that come out of your synthesizer. 
({\it Hint:} Sounds can be ``added".)\\
\item  Graph the  function $\colvec{3\\5}\in \Re^{\{1,2\}}$ .
\item Let $B=(\sin(t), \sin(2t))$. 
Explain why $\colvec{3\\5}_B$ is not in $\Re^{\{1,2\}}$ but is still a function.
\item Graph the function $\colvec{3\\5}_B$.
\end{enumerate}

\item
\begin{enumerate}
\item
Find the matrix for $\frac{d}{dx}$ acting on the vector space $V$ of polynomials of degree 2 or less in  the ordered basis $B=(x^2,x,1)$\\
\item Use the matrix from part (a) to rewrite the differential equation $\frac{d}{dx} p(x)=x$ as a matrix equation. Find all solutions of the matrix equation. Translate them into elements of $V$.\\\item Find  the matrix for $\frac{d}{dx}$ acting on the vector space $V$  in the ordered basis  $B'=(x^2+x,x^2-x,1)$.\\ 
\item  Use the matrix from part (c) to rewrite the differential equation $\frac{d}{dx} p(x)=x$ as a matrix equation. Find all solutions of the matrix equation. Translate them into elements of $V$.\\
\item Compare and contrast your results from parts (b) and (d).
\end{enumerate}

%%%%%%%%%%%%%%%%%%%
\item Find the ``matrix'' for $\frac{d}{dx}$ acting on the vector space of all power series in the ordered basis $(1,x,x^2,x^3,...)$. Use this matrix to find all power series solutions to the differential equation $\frac{d}{dx} f(x)=x$. {\it Hint:} your ``matrix'' may not have finite size.\\


\item Find the matrix for $\frac{d^2}{dx^2}$ acting on 
$\{ c_1 \cos(x)+c_2 \sin(x)  ~|~c_1,c_2\in \Re\}$ in the ordered basis $(\cos(x),\sin(x))$.\\

\item Find the matrix for $\frac{d}{dx}$ acting on $\{ c_1 \cosh(x)+c_2 \sinh(x) |c_1,c_2\in \Re\}$ in the ordered basis 
$$(\cosh(x),\sinh(x))$$ 
and in the ordered basis  $$(\cosh(x)+\sinh(x), \cosh(x)-\sinh(x)).$$\\
%(Recall that the hyperbolic trigonometric functions are defined by\\ 
%$\cosh(x)=\frac{e^x+e^{-x}}{2}, \sinh(x)=\frac{e^x-e^{-x}}{2}$.)

\item Let $B=(1,x,x^2)$ be an ordered basis for
$$V=\{ a_0+a_1x+a_2x^2~|~ a_0,a_1,a_2 \in \Re\}\, ,$$ 
and let 
$B'=(x^3,x^2,x,1)$ be an ordered basis for 
$$W=\{ a_0+a_1x+a_2x^2+a_3x^3 ~|~  a_0,a_1,a_2,a_3 \in \Re\}\, ,$$  
Find the matrix for the operator ${\cal I}:V\to W$  defined by $${\cal I}p(x)=\int_1^x p(t)dt$$ relative to these bases.


\item This exercise is meant to show you a generalization of the procedure you learned long ago for finding the function $mx+b$  given two points on its graph. It will also show you a way to think of matrices as members of a much bigger class of arrays of numbers. \\

Find the
\begin{enumerate}
\item constant function $f:\R \to \R $ whose graph contains $(2,3)$.
\item linear function $h:\R \to \R$ whose graph contains $(5,4)$.
\item first order polynomial function $g:\R \to \R$ whose graph contains 
$(1,2)$ and $(3,3)$.
\item second order polynomial function $p:\R \to \R$ whose graph contains $(1,0)$, $(3,0)$ and $(5,0)$.
\item second order polynomial function $q:\R \to \R$ whose graph contains $(1,1)$, $(3,2)$ and $(5,7)$.
\item second order homogeneous polynomial function $r:\R \to \R$ whose graph contains $(3,2)$.\\
\item \label{3rdPoly} number of points required to specify a third order polynomial $\R\to \R$.
\item number of points required to specify a third order homogeneous polynomial $\R\to \R$.
\item number of points required to specify a n-th order polynomial $\R\to \R$.
\item number of points required to specify a n-th order homogeneous polynomial $\R\to \R$.\\
\item first order polynomial function $F:\R^2 \to \R $ whose graph contains 
$\left(  \bv 0\\0 \ev , 1\right)$, 
$\left(  \bv 0\\1 \ev , 2\right)$, 
$\left(  \bv 1\\0 \ev , 3\right)$,  and
$\left(  \bv 1\\1 \ev , 4\right)$.

\item homogeneous first order polynomial function $H:\R^2 \to \R $ whose graph contains 
%$\left(  \bv 0\\0 \ev , 1\right)$, 
$\left(  \bv 0\\1 \ev , 2\right)$, 
$\left(  \bv 1\\0 \ev , 3\right)$,  and
$\left(  \bv 1\\1 \ev , 4\right)$.

\item second order polynomial function $J:\R^2 \to \R $ whose graph contains 
% ax^2 +by^2 +cxy + dx+ey+f
$\left(  \bv 0\\0 \ev , 0\right)$,  %f=0
$\left(  \bv 0\\1 \ev , 2\right)$, %b+e=2
$\left(  \bv 0\\2 \ev , 5\right)$, \\[.2cm]
$\left(  \bv 1\\0 \ev , 3\right)$,   %a=3
$\left(  \bv 2\\0 \ev , 6\right)$,  and %a=3
$\left(  \bv 1\\1 \ev , 4\right)$. %a+b

\item  first order polynomial function $K:\R^2 \to \R^2 $ whose graph contains 
$\left(  \bv 0\\0 \ev , \bv 1\\1 \ev \right)$, 
$\left(  \bv 0\\1 \ev , \bv 2\\2 \ev\right)$, \\[.2cm]
$\left(  \bv 1\\0 \ev , \bv 3\\3 \ev  \right)$,  and
$\left(  \bv 1\\1 \ev , \bv 4\\4 \ev\right)$.
\\

\item \label{PolyArray}How many points in the graph of a $q$-th order polynomial function $\R^n \to \R^n$ would completely determine the function? 


\item In particular, how many points of the graph of linear function $\R^n \to \R^n$ would completely determine the function? How does a matrix (in the standard basis) encode this information?

\item Propose a way to store the information required in \ref{3rdPoly} above in an array of numbers.

\item Propose a way to store the information required in \ref{PolyArray} above in an array of numbers.
\end{enumerate}






\end{enumerate}

\phantomnewpage
