
\subsection*{Hint for Review Question~\ref{mat_prob}}

%%%Insert this to get the typewriter font so it looks like a real movie script
{\ttfamily
\fontdimen2\font=0.4em
\fontdimen3\font=0.2em
\fontdimen4\font=0.1em
\fontdimen7\font=0.1em
\hyphenchar\font=`\-


\hypertarget{scripts_matrices_hint}{The majority of} the problem comes down to showing that matrices are right distributive. Let $M_k$ is all $n \times k$ matrices for any $n$, and define the map $f_R \colon M_k \rightarrow M_m$ by $f_R(M) = MR$ where $R$ is some $k \times m$ matrix. It should be clear that $f_R(\alpha \cdot M) = (\alpha M)R = \alpha (MR) = \alpha f_R(M)$ for any scalar $\alpha$. Now all that needs to be proved is that
\[
f_R(M + N) = (M + N)R = MR + NR = f_R(M) + f_R(N),
\]
and you can show this by looking at each entry.

\hypertarget{action}{We can actually generalize} the concept of this problem. Let $V$ be some vector space and $\mathbb{M}$ be some collection of matrices, and we say that $\mathbb{M}$ is a \emph{left-action}\index{Action} on $V$ if
\[
(M \cdot N) \circ v = M \circ (N \circ v)
\]
for all $M, N \in \mathbb{N}$ and $v \in V$ where $\cdot$ denoted multiplication in $\mathbb{M}$ (i.e. standard matrix multiplication) and $\circ$ denotes the matrix is a linear map on a vector (i.e. $M(v)$). There is a corresponding notion of a right action where
\[
v \circ (M \cdot N) = (v \circ M) \circ N
\]
where we treat $v \circ M$ as $M(v)$ as before, and note the order in which the matrices are applied. People will often omit the left or right because they are essentially the same, and just say that $\mathbb{M}$ acts on $V$.


} % Closing braket for font

%\newpage
