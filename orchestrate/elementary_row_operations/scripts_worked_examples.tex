
\subsection*{Worked Examples of Gaussian Elimination}

{\ttfamily
\fontdimen2\font=0.4em
\fontdimen3\font=0.2em
\fontdimen4\font=0.1em
\fontdimen7\font=0.1em
\hyphenchar\font=`\-

\hypertarget{scripts_elementary_row_operations_worked_examples}{Let us consider that we are} given two systems of equations that give rise to the following two (augmented) matrices:
\begin{align*}
\begin{pmatrix}
2 & 5 & 2 & 0 & \vline & 2 \\
1 & 1 & 1 & 0 & \vline & 1 \\
1 & 4 & 1 & 0 & \vline & 1
\end{pmatrix}
\quad\quad
\begin{pmatrix}
5 & 2 & \vline & 9 \\
0 & 5 & \vline & 10 \\
0 & 3 & \vline & 6
\end{pmatrix}
\end{align*}
and we want to find the solution to those systems. We will do so by doing Gaussian elimination.

For the first matrix we have
\begin{align*}
\begin{pmatrix}
2 & 5 & 2 & 0 & \vline & 2 \\
1 & 1 & 1 & 0 & \vline & 1 \\
1 & 4 & 1 & 0 & \vline & 1
\end{pmatrix}
\overset{R_1 \leftrightarrow R_2}{\sim} &
\begin{pmatrix}
1 & 1 & 1 & 0 & \vline & 1 \\
2 & 5 & 2 & 0 & \vline & 2 \\
1 & 4 & 1 & 0 & \vline & 1
\end{pmatrix}
\\ \overset{R_2 - 2 R_1 ; R_3 - R_1}{\sim} &
\begin{pmatrix}
1 & 1 & 1 & 0 & \vline & 1 \\
0 & 3 & 0 & 0 & \vline & 0 \\
0 & 3 & 0 & 0 & \vline & 0
\end{pmatrix}
\\ \overset{\frac{1}{3}R_2}{\sim} &
\begin{pmatrix}
1 & 1 & 1 & 0 & \vline & 1 \\
0 & 1 & 0 & 0 & \vline & 0 \\
0 & 3 & 0 & 0 & \vline & 0
\end{pmatrix}
\\ \overset{R_1 - R_2 ; R_3 - 3 R_2}{\sim} &
\begin{pmatrix}
1 & 0 & 1 & 0 & \vline & 1 \\
0 & 1 & 0 & 0 & \vline & 0 \\
0 & 0 & 0 & 0 & \vline & 0
\end{pmatrix}
\end{align*}
\begin{enumerate}[1.]
\item We begin by interchanging the first two rows in order to get a 1 in the upper-left hand corner and avoiding dealing with fractions.

\item Next we subtract row 1 from row 3 and twice from row 2 to get zeros in the left-most column.

\item Then we scale row 2 to have a 1 in the eventual pivot.

\item Finally we subtract row 2 from row 1 and three times from row 2 to get it into Reduced Row  Echelon Form.
\end{enumerate}
Therefore we can write $x = 1 - \lambda$, $y = 0$, $z = \lambda$ and $w = \mu$, or in vector form
\[
\begin{pmatrix}x\\y\\z\\w\end{pmatrix} = \begin{pmatrix}1\\0\\0\\0\end{pmatrix} + \lambda \begin{pmatrix}-1\\0\\1\\0\end{pmatrix} + \mu \begin{pmatrix}0\\0\\0\\1\end{pmatrix}.
\]

Now for the second system we have
\begin{align*}
\begin{pmatrix}
5 & 2 & \vline & 9 \\
0 & 5 & \vline & 10 \\
0 & 3 & \vline & 6
\end{pmatrix}
\overset{\frac{1}{5}R_2}{\sim} &
\begin{pmatrix}
5 & 2 & \vline & 9 \\
0 & 1 & \vline & 2 \\
0 & 3 & \vline & 6
\end{pmatrix}
\\ \overset{R_3 - 3 R_2}{\sim} &
\begin{pmatrix}
5 & 2 & \vline & 9 \\
0 & 1 & \vline & 2 \\
0 & 0 & \vline & 0
\end{pmatrix}
\\ \overset{R_1 - 2 R_2}{\sim} &
\begin{pmatrix}
5 & 0 & \vline & 5 \\
0 & 1 & \vline & 2 \\
0 & 0 & \vline & 0
\end{pmatrix}
\\ \overset{\frac{1}{5}R_1}{\sim} &
\begin{pmatrix}
1 & 0 & \vline & 1 \\
0 & 1 & \vline & 2 \\
0 & 0 & \vline & 0
\end{pmatrix}
\end{align*}
We scale the second and third rows appropriately in order to avoid fractions, then subtract the corresponding rows as before. Finally scale the first row and hence we have $x = 1$ and $y = 2$ as a unique solution.

} % Closing brace for the font

%\newpage
