\subsection*{Hint for Review Problem~\ref{QRprob}}

%%%Insert this to get the typewriter font so it looks like a real movie script
{\ttfamily
\fontdimen2\font=0.4em
\fontdimen3\font=0.2em
\fontdimen4\font=0.1em
\fontdimen7\font=0.1em
\hyphenchar\font=`\-


%%%%put a hypertarget around the opening bit of text
\hypertarget{gram_schmidt_and_orthogonal_complements_hint}{This} video shows you a way to solve problem~\ref{QRprob} 
that's different to the \hyperlink{methodQR}{method} described in the Lecture. The first thing is to think of 
$$
M=\begin{pmatrix}1 & 0 & 2 \\ -1 & 2 & 0 \\ -1 & 2 \ & 2\end{pmatrix}
$$
as a set of 3 vectors 
$$
v_1=\begin{pmatrix}0 \\ -1 \\ -1\end{pmatrix}\, ,\qquad
v_2=\begin{pmatrix}0 \\ 2 \\ -2\end{pmatrix}\, ,\qquad
v_3=\begin{pmatrix}2 \\ 0 \\ 2\end{pmatrix}\, .
$$
Then you need to remember that we are searching for a decomposition
$$M=QR$$ 
where $Q$ is an orthogonal matrix. Thus the upper triangular matrix $R = Q^T M$ and $Q^T Q = I$.
Moreover, orthogonal matrices perform rotations. To see this compare the inner product $u\dotprod v = u^T v$ of vectors $u$ and $v$
with that of $Qu$ and $Qv$:
$$
(Qu)\dotprod (Qv) = (Qu)^T (Qv) = u^T Q^T Q v = u^T v = u\dotprod v\, .
$$  
Since the dot product doesn't change, we learn that $Q$ does not change angles or lengths of vectors.

Now, here's an interesting procedure: rotate $v_1, v_2$ and $v_3$ such that $v_1$ is along the $x$-axis, $v_2$ is in the $xy$-plane.
Then if you put these in a matrix you get something of the form
$$
\begin{pmatrix}a & b & c \\ 0 & d & e \\ 0  & 0 & f\end{pmatrix}
$$
which is exactly what we want for $R$!

Moreover, the vector 
$$
\begin{pmatrix}
a \\ 0 \\ 0
\end{pmatrix}
$$
is the rotated $v_1$ so must have length $||v_1|| = \sqrt{3}$. Thus $a= \sqrt{3}$. 

The rotated $v_2$ is
$$
\begin{pmatrix}
b \\ d \\ 0
\end{pmatrix}
$$
and must have length $||v_2||=2\sqrt{2}$. Also the dot product between  
$$
\begin{pmatrix}
a \\ 0 \\ 0
\end{pmatrix}
\mbox{ and }
\begin{pmatrix}
b \\ d \\ 0
\end{pmatrix}
$$
is $ab$ and
must equal $v_1\dotprod v_2=0$. (That $v_1$ and $v_2$ were orthogonal is just a coincidence here... .) Thus $b=0$.
So now we know most of the matrix $R$
$$
R=\begin{pmatrix}\sqrt{3} & 0 & c \\ 0 & 2\sqrt{2} & e \\ 0  & 0 & f\end{pmatrix}\, .
$$
You can work out the last column using the same ideas. Thus it only remains to compute $Q$ from
$$
Q=M R^{-1}\, .
$$

 
%%%%don't forget to close the bracket so the stuff after your file doesn't look like a movie!
}

%\newpage