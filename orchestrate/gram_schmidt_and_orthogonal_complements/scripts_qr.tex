\subsection*{Another $QR$ Decomposition Example}

%%%Insert this to get the typewriter font so it looks like a real movie script
{\ttfamily
\fontdimen2\font=0.4em
\fontdimen3\font=0.2em
\fontdimen4\font=0.1em
\fontdimen7\font=0.1em
\hyphenchar\font=`\-


%%%%put a hypertarget around the opening bit of text
\hypertarget{scripts_gram_schmidt_and_orthogonal_complements_qr_example}{We can alternatively think of the} $QR$ decomposition as performing the Gram-Schmidt procedure on the \emph{column space}, the vector space of the column vectors of the matrix, of the matrix $M$. The resulting orthonormal basis will be stored in $Q$ and the negative of the coefficients will be recorded in $R$. Note that $R$ is upper triangular by how Gram-Schmidt works. Here we will explicitly do an example with the matrix
\[
M = \begin{pmatrix}
\mc\vline & \mc\vline & \mc\vline \\
m_1 & m_2 & m_3 \\
\mc\vline & \mc\vline & \mc\vline
\end{pmatrix} = \begin{pmatrix}
1 & 1 & -1 \\
0 & 1 & 2 \\
-1 & 1 & 1
\end{pmatrix}.
\]
First we normalize $m_1$ to get $m_1^{\prime} = \frac{m_1}{\norm{m_1}}$ where $\norm{m_1} = r_1^1 = \sqrt{2}$ which gives the decomposition
\[
Q_1 = \begin{pmatrix}
\frac{1}{\sqrt{2}} & 1 & -1 \\
\mc 0 & 1 & 2 \\
-\frac{1}{\sqrt{2}} & 1 & 1
\end{pmatrix},
\hspace{20pt}
R_1 = \begin{pmatrix}
\sqrt{2} & 0 & 0 \\
0 & 1 & 0 \\
0 & 0 & 1
\end{pmatrix}.
\]
Next we find
\[
t_2 = m_2 - (m_1^{\prime} \dotprod m_2) m_1^{\prime} = m_2 - r^1_2 m^{\prime}_1 = m_2 - 0 m_1^{\prime}
\]
noting that
\[
m_1^{\prime} \dotprod m_1^{\prime} = \norm{m_1^{\prime}}^2 = 1
\]
and $\norm{t_2} = r^2_2 = \sqrt{3}$, and so we get $m_2^{\prime} = \frac{t_2}{\norm{t_2}}$ with the decomposition
\[
Q_2 = \begin{pmatrix}
\frac{1}{\sqrt{2}} & \frac{1}{\sqrt{3}} & -1 \\
\mc0 & \frac{1}{\sqrt{3}} & 2 \\
-\frac{1}{\sqrt{2}} & \frac{1}{\sqrt{3}} & 1
\end{pmatrix},
\hspace{20pt}
R_2 = \begin{pmatrix}
\sqrt{2} & 0 & 0\\
0 & \sqrt{3} & 0 \\
0 & 0 & 1
\end{pmatrix}.
\]
Finally we calculate
\begin{align*}
t_3 & = m_3 - (m_1^{\prime} \dotprod m_3) m_1^{\prime} - (m_2^{\prime} \dotprod m_3) m_2^{\prime}
\\ & = m_3 - r^1_3 m_1^{\prime} - r^2_3 m_2^{\prime} = m_3 + \sqrt{2} m_1^{\prime} - \frac{2}{\sqrt{3}} m_2^{\prime},
\end{align*}
again noting $m_2^{\prime} \dotprod m_2^{\prime} = \norm{m_2^{\prime}} = 1$, and let $m_3^{\prime} = \frac{t_3}{\norm{t_3}}$ where $\norm{t_3} = r^3_3 = 2 \sqrt{\frac{2}{3}}$. Thus we get our final $M = QR$ decomposition as
\[
Q = \begin{pmatrix}
\frac{1}{\sqrt{2}} & \frac{1}{\sqrt{3}} & - \frac{1}{\sqrt{2}} \\
\mc0 & \frac{1}{\sqrt{3}} & \sqrt{\frac{2}{3}} \\
-\frac{1}{\sqrt{2}} & \frac{1}{3} & - \frac{1}{\sqrt{6}}
\end{pmatrix},
\hspace{20pt}
R = \begin{pmatrix}
\sqrt{2} & 0 & -\sqrt{2} \\
0 & \sqrt{3} & \frac{2}{\sqrt{3}} \\
0 & 0 & 2 \sqrt{\frac{2}{3}}
\end{pmatrix}.
\]

} % Closing brace

%\newpage