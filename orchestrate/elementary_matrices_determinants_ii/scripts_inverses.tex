
\subsection*{Determinants and Inverses}

%%%Insert this to get the typewriter font so it looks like a real movie script
{\ttfamily
\fontdimen2\font=0.4em
\fontdimen3\font=0.2em
\fontdimen4\font=0.1em
\fontdimen7\font=0.1em
\hyphenchar\font=`\-


\hypertarget{scripts_elementary_matrices_determinants_ii_inverses}{Lets figure out} the relationship between determinants and invertibility. If we have a system of equations $Mx=b$ and we have the inverse $M^{-1}$ then if we multiply on both sides we get $ x = M^{-1}Mx= M^{-1}b$. If the inverse exists we can solve for $x$ and get a solution that looks like a point. 

So what could go wrong when we want solve a system of equations and get a solution that looks like a point? Something would go wrong if we didn't have enough equations for example if we were just given 
\[
x+y = 1
\]
or maybe, to make this a square matrix $M$ we could write this as 
\begin{align*}
x+y &= 1\\
0 &= 0
\end{align*}
The matrix for this would be 
$M =\begin{bmatrix}
1 & 1\\
0& 0 
\end{bmatrix}$ 
and det$(M) = 0$. When we compute the determinant, this row of all zeros gets multiplied in every term. If instead we were given redundant equations 

\begin{align*}
x+y &= 1\\
2x+2y &= 2
\end{align*}
The matrix for this would be 
$M =\begin{bmatrix}
1 & 1\\
2& 2 
\end{bmatrix}$  and det$(M) = 0$. But we know that with an elementary row operation, we could replace the second row with a row of all zeros. Somehow the determinant is able to detect that there is only one equation here. Even if we had a set of contradictory set of equations such as
\begin{align*}
x+y &= 1\\
2x+2y &= 0,
\end{align*}
where it is not possible for both of these equations to be true, the matrix $M$ is still the same, and still has a determinant zero.

Lets look at a three by three example, where the third equation is the sum of the first two equations.

\begin{align*}
x+y + z &= 1\\
y +z &= 1\\
x + 2y+ 2z &= 2\\
\end{align*}

and the matrix for this is 

\[
M =\begin{bmatrix}
1 & 1 &1\\
0 & 1 & 1\\
1 & 2& 2 
\end{bmatrix}
\]

If we were trying to find the inverse to this matrix using elementary matrices
$$ \left( \begin{array}{ccc | ccc}
1 & 1 &1 & 	1 & 0 & 0\\
0 & 1 & 1 & 	0 & 1 & 0 \\
1 & 2 & 2 &	0 & 0 & 1
\end{array} \right) 
=
 \left( \begin{array}{ccc | rrr}
1 & 1 &1 & 	1 & 0 & 0\\
0 & 1 & 1 & 	0 & 1 & 0 \\
0 & 0 & 0 &	-1 & -1 & 1
\end{array} \right) 
$$
And we would be stuck here. The last row of all zeros cannot be converted into the bottom row of a $3 \times 3$ identity matrix. this matrix has no inverse, and the row of all zeros ensures that the determinant will be zero. It can be difficult to see when one of the rows of a matrix is a linear combination of the others, and what makes the determinant a useful tool is that with this reasonably simple computation we can find out if the matrix is invertible, and if the system will have a solution of a single point or column vector. 
} % Closing bracket for font

%\newpage
