\href{\webworkurl}{Webwork}


\begin{enumerate}
\item Let $D=\begin{pmatrix}
\lambda_1 & 0 \\
0 & \lambda_2 \\
\end{pmatrix}$.
\begin{enumerate}
\item Write $D$ in terms of the vectors $e_1$ and $e_2$, and their transposes.
\item Suppose $P=\begin{pmatrix}
a & b \\
c & d \\
\end{pmatrix}$ is invertible.  Show that $D$ is similar to
\[
M=\frac{1}{ad-bc}\begin{pmatrix}
\lambda_1ad-\lambda_2bc & -(\lambda_1-\lambda_2)ab \\[1mm]
(\lambda_1-\lambda_2)cd & -\lambda_1bc + \lambda_2ad
\end{pmatrix}.
\]
\item Suppose the vectors $\rowvec{a&b}^T$ and $\rowvec{c&d}^T$ are orthogonal.  What can you say about $M$ in this case? (Hint: think about what \(M^T\) is equal to.)
\end{enumerate}

\phantomnewpage

\item \label{orthogprob} Suppose $S=\{v_1, \ldots, v_n \}$ is an \emph{orthogonal} (not orthonormal) basis for~$\Re^n$.  Then we can write any vector $v$ as $v=\sum_ic^iv_i$ for some constants $c^i$.  Find a formula for the constants $c^i$ in terms of $v$ and the vectors in~$S$.

\videoscriptlink{orthonormal_bases_hint.mp4}{Hint for ~\ref{orthogprob} }{scripts_orthonormal_bases_hint}
\phantomnewpage

\item \label{orthogprojprob} Let $u,v$ be linearly independent vectors in $\Re^3$, and $P=\spa \{ u,v\}$ be the plane spanned by $u$ and $v$.  
\begin{enumerate}
\item Is the vector $v^\bot := v-\frac{u\cdot v}{u\cdot u}u$ in the plane $P$?
\item  What is the (cosine of the) angle between $v^\bot$ and $u$?
\item %Given your solution to the above, 
How can you find a third vector perpendicular to both $u$ and $v^\bot$?
\item  Construct an orthonormal basis for $\Re^3$ from $u$ and $v$.
\item  Test your abstract formulae starting with 
\[
u=\rowvec{1 & 2 & 0} \text{ and } v=\rowvec{0 & 1 & 1}.
\]
\end{enumerate}

\videoscriptlink{orthonormal_bases_hint3.mp4}{Hint for ~\ref{orthogprojprob}}{scripts_orthonormal_bases_hint3}

\phantomnewpage



\item Find an orthonormal  basis for $\Re^4$ which includes $(1,1,1,1)^T$ through the following procedure:\\
\begin{enumerate} 
\item Pick a vector perpendicular to the vector $v_1$ defined by $v_1^T=(1,1,1,1)$ from the solution set of the matrix equation $v_1^Tx=0$. Pick the vector $v_2$ obtained from the standard procedure which is the coefficient of $x_2$.
\item Pick a vector perpendicular to both $v_1$ and $v_2$ from the solutions set of the matrix equation $\colvec{v_1^T\\v_2^T}x=0$. Pick the vector $v_3$ obtained from the standard procedure with $x_3$ as the coefficient. 
\item Pick a vector perpendicular to $v_1,v_2,$ and $v_3$ from the solution set of the matrix equation $\colvec{v_1^T\\v_2^T\\v_3^T}x=0$.  Pick the vector $v_4$ obtained from the standard procedure with $x_3$ as the coefficient. 
\item Normalize the four vectors obtained   above.
\end{enumerate}


\item Using the inner product $f\cdot g = \int_0^1 f(x)g(x)dx$  on the vector space ${\rm span} \{ 1,x,x^2,x^3\}$ is then perform the Gram-Schmidt procedure on the set $( 1,x,x^2,x^3)$. 

\item 
\begin{enumerate}
\item
Show that if $Q$ is an orthogonal $n\times n$ matrix then $u\cdot v = (Qu)\cdot (Qv)$ for any $u,v\in \Re^n$. That is, $Q$ preserves the inner product. 
\item Does $Q$ preserve the outer product? 
\item  If $\{ u_1,\dots,u_n\}$ is an orthonormal set and $\{ \lambda_1,\cdots,\lambda_n\}$ is a set of numbers 
then what are the eigenvalues and eigenvectors of the matrix
$M=\sum_{i=1}^n \lambda_i u_i u_i^T$? 
\item How does $Q$ change this matrix? How do the eigenvectors and eigenvalues change? 

\end{enumerate}












\end{enumerate}




























