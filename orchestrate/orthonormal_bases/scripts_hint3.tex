
\subsection*{Hint  for Review Problem~\ref{orthogprojprob}}

%%%Insert this to get the typewriter font so it looks like a real movie script
{\ttfamily
\fontdimen2\font=0.4em
\fontdimen3\font=0.2em
\fontdimen4\font=0.1em
\fontdimen7\font=0.1em
\hyphenchar\font=`\-


\hypertarget{scripts_orthonormal_bases_hint3}{Lets} 
work part by part:
\begin{enumerate}[(a)]
\item Is the vector $v^\bot = v-\frac{u\cdot v}{u\cdot u}u$ in the plane $P$? 

%This part will make more sense if you think back to the formula for projection formula from multivariable calculus. The projection of $\mathbf{v}$ in the direction of $\mathbf{u}$
%\[
%\text{proj}_{\mathbf{u}}(\mathbf{v}) = \frac{\mathbf{u}\cdot \mathbf{v}}{\mathbf{u}\cdot \mathbf{u}}\mathbf{u} %= \frac{\mathbf{u}\cdot \mathbf{v}}{\norm{\mathbf{u}}} \frac{\mathbf{u}}{\norm{\mathbf{u}}}
%\]
Remember that the dot product gives you a scalar not a vector, so if you think about this formula $\frac{\mathbf{u}\cdot \mathbf{v}}{\mathbf{u}\cdot \mathbf{u}}$ is a scalar, so this is a linear combination of $\mathbf{v}$ and $\mathbf{u}$. Do you think it is in the span?

\item What is the angle between $v^\bot$ and $u$?

This part will make more sense if you think back to the dot product formulas you probably first saw in multivariable calculus. Remember that 
\[\mathbf{u}\cdot \mathbf{v} = \norm{\mathbf{u}} \norm{\mathbf{v}} \cos(\theta),
\] 
and in particular if they are perpendicular $\theta = \frac{\pi}{2}$ and $\cos(\frac{\pi}{2}) = 0$ you will get $\mathbf{u}\cdot \mathbf{v} = 0$.

Now try to compute the dot product of $u$ and $v^\bot$ to find $ \norm{\mathbf{u}} \norm{\mathbf{v^\bot}} \cos(\theta)$

\begin{eqnarray*}
u\cdot v^\bot &=& u\cdot \left( v  -  \frac{u\cdot v}{u\cdot u}u \right) \\
&=& u\cdot  v  - u\cdot \left( \frac{u\cdot v}{u\cdot u} \right)u \\
&=& u\cdot  v  - \left( \frac{u\cdot v}{u\cdot u} \right) u\cdot u \\
\end{eqnarray*}

Now you finish simplifying and see if you can figure out what $\theta $ has to be.

\item Given your solution to the above, how can you find a third vector perpendicular to both $u$ and $v^\bot$?

Remember what other things you learned in multivariable calculus? This might be a good time to remind your self what the cross product does.

\item  Construct an orthonormal basis for $\Re^3$ from $u$ and $v$.

If you did part (c) you can probably find 3 orthogonal vectors to make a orthogonal basis. All you need to do to turn this into an orthonormal  basis is make these into unit vectors. 

\item  Test your abstract formulae starting with 
\[
u=\rowvec{1 & 2 & 0} \text{ and } v=\rowvec{0 & 1 & 1}.
\]

Try it out, and if you get stuck try drawing a sketch of the vectors you have.



\end{enumerate}


} % Closing bracket for font

%\newpage
